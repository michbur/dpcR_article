\documentclass{bioinfo}
\copyrightyear{2005}
\pubyear{2005}

\begin{document}
\firstpage{1}

\title[dpcR]{dpcrR: a Swiss-army knife for the analysis of digital PCR experiments}
\author[Burdukiewicz \textit{et~al.}]{Micha\l{} Burdukiewicz\,$^{1}$, Jim Huggett\,$^2$, Alexandra Whale\,$^2$, Bart K.M. Jacobs\,$^3$, Lieven Clement\,$^3$, Piotr Sobczyk\,$^{1}$, Andrej-Nikolai Spiess\,$^4$, Peter Schierack\,$^5$, and Stefan R\"odiger\,$^{5}$\footnote{to whom correspondence should be addressed}}
\address{$^{1}$Department of Genomics, Faculty of Biotechnology, University of 
Wroc\l{}aw, Wroc\l{}aw, Poland\\
$^{2}$Molecular and Cell Biology Team, LGC, Teddington, United Kingdom\\
$^{3}$Department of Applied Mathematics, Computer Science and Statistics, Ghent University, Belgium\\
$^{4}$University Medical Center Hamburg-Eppendorf, Hamburg, Germany\\
$^{5}$Faculty of Natural Sciences, Brandenburg University of Technology 
Cottbus--Senftenberg, Germany}

\history{Received on XXXXX; revised on XXXXX; accepted on XXXXX}

\editor{Associate Editor: XXXXXXX}

\maketitle

\begin{abstract}

\section{Motivation:}
The digital Polymerase Chain Reaction (dPCR) is emerging in all research areas, 
such as life-sciences and diagnostics. dPCR is likely to have the same impact in 
nucleic acids quantification as real-time quantitative PCR. Advantages over 
conventional qPCR include the possibility of absolute quantification and the 
drastically reduced sensitivity to inhibitors. There are different technical 
approaches (e.g., droplets, nano-structured chambers) and different approaches 
for statistical analysis were proposed. However, a unified open software package 
is not available.

\section{Methods:}
To cover all methods of dPCR we implemented all accessible peer-review methods 
and plots into the \textit{dpcR} \textbf{R} package with a plug-in like 
architecture. \textbf{R} is a sophisticated statistical computing environment. 
The software uses abstraction to make it usable for droplets and chamber based 
technologies.

\section{Results:}
\textit{dpcR} is versatile open source cross-platform software package, which 
provides functions to process dPCR data. Our software can be used for data 
analysis and presentation, as frame-work for novel technical developments and as 
reference for statistical methods in dPCR analysis.

 We implemented many functions with binding 
to the \textit{shiny} \textbf{R} package \cite{shiny} to provide means to run it as 
interactive web application. Features such as functions to estimate the 
underlying Poisson process, methods from peer-reviewed literature for 
calculating confidence intervals based on single samples as well as on 
replicates, a novel Generalized Linear Model-based procedure to compare digital 
PCR experiments and a spatial randomness test for assessing plate effects have 
been integrated. Thus, the \textit{dpcR} package can be used by\textbf{R} novices in a 
graphical user interface or on expert level in R. The \textit{dpcR} package is 
an open environment, which can be adopted to the growing knowledge in dPCR. The 
\textit{dpcR} package can be used to build a custom-made analyzer according to 
the wishes of the user.
\section{Conclusion:}

\section{Availability:}
http://cran.r-project.org/web/packages/dpcR\\newline
Source code: https://github.com/michbur/dpcR\

\section{Contact:} \href{stefan.roediger@b-tu.de}{stefan.roediger@b-tu.de}
\end{abstract}

\section{Introduction}
There are three principal approaches to quantify nucleic acids. The fist is by 
referencing the material to an external calibrator (qPCR), The standard approach 
to quantify nucleic acids has been the real-time quantitative PCR (qPCR) so far 
\cite{pabinger_survey_2014}. It is a well established and robust technology, 
which allows precise quantification of DNA material in high throughput fashion 
at a reasonable price. However, the quantification by qPCR is challenging at 
very low and very high concentrations. At low concentration Monte Carlo effect 
play a major role and at high concentration inhibition process start to dominate 
the qPCR . Thus, the qPCR is only usable in the working range of the calibrator. 
In addition, pre-processing and data analysis is a affected by numerous adverse 
effects \cite{spiess_impact_2015}. The second approach is to count the number of 
molecules (e.g., superSAGE or NanoStrings) (Matsumura-2006-Nature-Methods, 
Waggott-2012-Bioinformatics). The third is to analyze the number of positive 
reactions in relation to the number of total reaction (dPCR). Since 
approximately ten year the digital PCR (dPCR) is gaining entrance in the 
mainstream user-base. There is currently an intensive research on qPCR platforms 
with the overall aim to make to technology broadly usable, cheap, robust and to 
enable high sample throughput. The chemical basis of the dPCR is identical to 
the qPCR, which includes master-mix preparation and thermal cycling of the 
sample. In contrast to qPCR the amplification ration does not take place in a 
single reaction chamber but is rather a process of clonal amplification in small 
separate ``compartments'' (e.g., nl volume droplets of water oil emulsions, 
chambers on micro structured chips). The quantification of the amplification is 
not done by determining a Cq-value derived from an amplification curve but 
applying a Poisson distribution based determination of the concentration of the 
starting material. Therefore, the dPCR does not require an external calibration 
\cite{selck_increased_2013, rodiger_r_2015}.

A first proposal for digital PCR like approach and the use of the Poisson 
distribution to quantify the number of molecules on a ``sample'' was shown by 
Ruano et al. 1990 (PNAS) with the single molecule dilution (SMD) PCR. In 1999 
Vogelstein et al. (PNAS) described the first true digital PCR \cite{morley_digital_2014}. Application of 
the dPCR cover all applications of conventional qPCR, including investigation of 
alleles, gene expression analysis and absolute quantification of PCR products. 
For absolute quantification the qPCR relied on an external calibrator 
(calibration curve) which was derived serial decadic dilution (e.g., 1:10 $\rightarrow$ 
1:100 $\rightarrow$ 1:1000) of a known target input quantity. The real-time monitoring of 
the PCR product formation enabled to determine quantification points (Cq). The 
Cq are strictly related to the input quantity. A simple arithmetic operation 
(after logarithmic transformation of the concentration) is sufficient to 
determine any nucleic acid quantity \cite{huggett_considerations_2014}.

qPCR	dPCR
Number of copies/DNA per volume (e.g., ng/µl, copies/µl)	total number of compartments * ln (...)

The dPCR has some principle assumptions and fundamental properties. First of all 
the chemical reaction should be not affected by inhibitors. The distribution of 
the single molecule target regions follows a Poission distribution. The Poisson 
distribution appears like a normal distribution but without negative values and 
being zero the lowest. First a large number (n) of amplifications reactions as 
required to have a high statistical power. Therefore in practical terms a 
massive number of PCR reactions is needed. For Poission distributions an n of XY 
(get reference from table/text book form statistics/biostatistics?) is 
considered large. Second that the molecules required for the amplification 
amplifications reactions are randomly distributed in the compartments. Visual 
analysis, Ripley's K functions or ??? can be used to test for randomness of the 
reaction and thus to exclude the clustering of of positive reactions. A 
clustering of positive wells might be due to sample loading or analysis process 
(systematical error). The outcome of an amplification can be no amplification at 
all (less than 1 copy per volume), an unsaturated reaction with a 
binary/``multinary'' amplification (usable to calculate the ``concentration'') 
or a saturated reaction where virtually all compartments are positive.

Calculation of the ``Concentration''
Reference to ``Supplement''

Calculation of the uncertainty
To determine the uncertainty of the calculations two approach have been proposed 
in the peer-review literature (Dube 2008, PLoS One, Bath …). The uncertainty is 
dependent on the number of PCR reactions (reference to \textit{\textit{dpcR}} 
functions). … Reference to ``Supplement'' and \textit{dpcR} functions.

We developed the \textit{dpcR} package which is software suite for analysis of 
dPCR based on the open source statistical software R. The \textit{dpcR} includes 
to invitation to the scientific community to join and support the development of 
\textit{dpcR} (github?). The aim of the software is to provide the scientific 
community a tool for teaching purposes, data analysis, theoretical research 
(simulation) and to accelerate the development of new approaches to dPCR. We 
implemented all existing statistical methods for dPCR and suggest the 
introduction of a standardized nomenclature for qPCR. The package enables the 
simulations and predictions of Poisson distribution for dPCR scenarios, the 
analysis of previously run dPCRs.

Interactive use and graphical representation with \textit{shiny} \cite{shiny}.

Import and export of results figures and data.

There are currently two technical approaches to dPCR. dPCRs may use		%  (e.g., QuantStudio$\circledR$ 12K Flex (Life Technologies), BioMark\texttrademarkTM  EP1TM (Fluidigm))
(microfluidic)chambers or emulsion based chambers 
(QX200 \texttrademark (Bio-Rad), RainDrop \texttrademark System (RainDance)). 
Chamber based dPCR systems have fixed geometries, including the volume of the 
reaction chambers. Despite the fact that dPCRs is an endpoint analysis the 
chamber based technologies allow generally the real-time monitoring of the 
amplification reaction and subsequent confirmation of the amplification reaction 
be melting curve analysis. Thus, such technologies enable easier trouble 
shooting and quality management of the data. However, the downside of these 
technologies is the fixed limited number of compartments and the price. The 
emulsion based dPCRs are easier to perform since the compartments are generated 
by microfluidic technologies and have practically no limitation regarding the 
number of compartments. This results in a higher statistical power to quantify 
small differences in sample quantities. The emulsion chambers are made of 
water-in-oil emulsions with similar sizes.


Two-sided exact tests and matching confidence intervals for discrete data \cite{fay_2010}

Controlling the False Discovery Rate: A Practical and Powerful Approach to Multiple Testing \cite{benjamini_1995}

Interval Estimation for a Binomial Proportion \cite{brown_2001}



We have chosen \textbf{R} because it is the \textit{lingua franca} in 
biostatistics and broadly used in other disciplines \cite{rodiger_r_2015}. The 
are many packages in existence which enable the fast development of new methods 
and plotting facilities. As most \textbf{R} packages depend on one or more other 
packages \cite{ooms_2013} depends \textit{dpcR} on other packages, resulting in 
a complex network of recursive dependencies. Core packages \textit{qpcR} 
\cite{ritz_qpcr_2008}, \textit{shiny} \cite{shiny}, \textit{MBmca} 
\cite{rodiger_surface_2013}, \textit{chipPCR} \cite{rodiger_chippcr_2015} and 
further packages as shown in the dependency graph (Supplement XYZ).

 

\textbf{R} has a rich set of tool to arrange data (reshape?) in order to prepare them for 
the analysis. This is important when it comes to the question how experiments 
should be treated. It is possible to analyze the PCR reaction the panels 
independently (effect on CI and uncertainty) or to pool/aggregate all reactions 
(effect on CI and uncertainty) to achieve higher sensitivity/certainty.


\begin{methods}
\section{Implementation}

The source code is open source (GPL-3 or later) and hosted at \textit{github.com}. The stable package is freely 
available from Bioconductor.

One basic design decision was to structure specific properties of digital PCR 
systems (dropet vs. chamber) in auxiliary functions and to perform central 
calculation specific to Poisson statistics in independent main functions. 
Chamber digital PCR systems fundamentally rely on the proper preprocessing of 
qPCR data. We have chosen to implement the core functionality by a dependency to 
the \textit{qpcR} \textbf{R} package \cite{ritz_qpcr_2008}. The main functions (e.g., for analysis, 
simulations, plotting), several auxiliary helper functions (e.g., data import) 
and data set of different dPCR systems are listed in Table XY. Further 
dependencies to 3rd party packages include \textit{pracma}, ... . See the vignette for 
details.

The GUI employs advanced plots based on \textit{ggplot2} \cite{kahle_wickham_2013}.



\end{methods}


\section{Discussion}

Functions included may be used to simulate dPCRs, perform statistical data 
analysis, plotting of the results and simple report generation. 

There are currently different software solutions for dPCR analysis such as the 
OpenArray software (Life Technologies) or XYZ (Bio-Rad). Most of the are black 
boxes which prevent deep insight into the data processing step. In addition most 
of the software solutions are aimed to be used in very specific scenarios and a 
mutual exclusive to alternative platforms (e.g., droplet vs. chamber-based). We 
have chosen \textbf{R} because it is the \textit{lingua franca} in biostatistics and broadly used 
in other disciplines \cite{rodiger_r_2015}.



%%%%%%%%%%%%%%%%%%%%%%%%%%%%%%%%%%%%%%%%%%%%%%%%%%%%%%%%%%%%%%%%%%%%%%%%%%%%%%%%%%%%%
%
%     please remove the " % " symbol from \centerline{\includegraphics{fig01.eps}}
%     as it may ignore the figures.
%
%%%%%%%%%%%%%%%%%%%%%%%%%%%%%%%%%%%%%%%%%%%%%%%%%%%%%%%%%%%%%%%%%%%%%%%%%%%%%%%%%%%%%%






\section{Conclusion}

In conclusion, \textit{dpcR} provides means to understand how digital PCR works, 
to design, simulate and analyze experiments, and to verify their results (e.g., 
confidence interval estimation), which should ultimately improve 
reproducibility. We have built what we believe to be the first unified, 
cross-platform, dMIQE compliant, open source (GPL-3 or later) software frame-work for 
analyzing digital PCR experiments. Our frame-work, designated \textit{dpcR} , is 
targeted at a broad user base including end users in clinics, academics, 
developers, and educators. We implemented existing statistical methods for dPCR 
and suggest the introduction of a standardized nomenclature for qPCR. Our 
frame-work is suitable for teaching and includes references for an elaborated 
set of methods for dPCR statistics. Our software can be used for (I) data 
analysis and visualization in research, (II) as software frame-work for novel 
technical developments, (III) as platform for teaching this new technology and 
(IV) as reference for statistical methods with a standardized nomenclature for 
dPCR experiments. The package enables the simulations and predictions of Poisson 
distribution for dPCR scenarios, the analysis of previously run dPCRs. Due to 
the plug-in structure of the software it is possible to build custom-made 
analyzers.

\section*{Acknowledgement}

Grateful thanks belong to the \textbf{R} community.

\paragraph{Funding\textcolon} This work was funded by the Federal Ministry of Education and Research (BMBF)
 InnoProfile--Transfer--Projekt 03 IPT 611X.

\paragraph{Conflict of Interest\textcolon} none declared.

%\bibliographystyle{natbib}
%\bibliographystyle{achemnat}
%\bibliographystyle{plainnat}
%\bibliographystyle{abbrv}
%\bibliographystyle{bioinformatics}
%
\bibliographystyle{plain}
%
\bibliography{dpcr}

\end{document}
